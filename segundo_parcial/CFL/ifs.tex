\documentclass[11pt]{article}

\usepackage[spanish]{babel}
\usepackage[utf8]{inputenc}
\usepackage[T1]{fontenc}

\usepackage{charter} %tipo de fuente

\usepackage{textcomp} %Paquete para algunos caracteres especiales


\usepackage{amsmath, amsfonts, amssymb}

\usepackage[left=1.5cm, right=1.5cm, top=2.5cm]{geometry}

%para escribir codigo
\usepackage{listings}


\usepackage{float}


\title{Gramáticas Libres de Contexto, notación BNF: \textit{If's anidados}}
\author{Ian Mendoza Jaimes}


\begin{document}

\maketitle

\begin{center}

Teoría Computacional\\

Grupo: 2CM4\\

Profesor: Genaro Juárez Martínez\\
\end{center}

\newpage


\section{Definición del problema}
Las GFC son un tipo de gramática formal de la forma: $V \rightarrow W$, donde V es un símbolo no terminal y w es una cadena de terminales y/o no terminales. El término libre de contexto se refiere al hecho de que el no terminal V puede siempre ser sustituido por w sin tener en cuenta el contexto en el que ocurra.\\

Para entender mejor este concepto se realizo el siguiente ejercicio: Construir 10 if's anidados usando CFG con la notación BNF. Con esta información podemos definir la siguiente estructura de nuestro if: \[ <statement> ::= if <condition> then <statement> [; else <statement>] \] con las siguientes reglas de producción: 

\begin{equation}
S \rightarrow iCtSA
\end{equation}

\begin{equation}
A \rightarrow ;eS | \varepsilon
\end{equation}

\vspace{1em}

\section{Producción de 10 if's anidados}

\subsection{Usando las reglas de generación}

\begin{enumerate}

\item Utilizando (1): \[ S \rightarrow iCtSA \]
\item Utilizando (2): \[ iCtSA \rightarrow iCtS;eS \]
\item Utilizando (1): \[ iCtS;eS \rightarrow iCtS;eiCtSA \]
\item Utilizando (2): \[ iCtS;eiCtSA \rightarrow iCtS;eiCtS\varepsilon \]
\item Utilizando (1): \[ iCtS;eiCtS\varepsilon \rightarrow iCtS;eiCtiCtSA\varepsilon \]
\item Utilizando (2): \[ iCtS;eiCtiCtSA\varepsilon \rightarrow iCtS;eiCtiCtS\varepsilon\varepsilon \]
\item Utilizando (1): \[ iCtS;eiCtiCtS\varepsilon\varepsilon \rightarrow iCtiCtSA;eiCtiCtS\varepsilon\varepsilon \]
\item Utilizando (2): \[ iCtiCtSA;eiCtiCtS\varepsilon\varepsilon \rightarrow iCtiCtS\varepsilon;eiCtiCtS\varepsilon\varepsilon \]
\item Utilizando (1): \[ iCtiCtS\varepsilon;eiCtiCtS\varepsilon\varepsilon \rightarrow iCtiCtiCtSA\varepsilon;eiCtiCtS\varepsilon\varepsilon \]
\item Utilizando (2): \[ iCtiCtiCtSA\varepsilon;eiCtiCtS\varepsilon\varepsilon \rightarrow  iCtiCtiCtS\varepsilon\varepsilon;eiCtiCtS\varepsilon\varepsilon \]
\item Utilizando (1): \[ iCtiCtiCtS\varepsilon\varepsilon;eiCtiCtS\varepsilon\varepsilon \rightarrow iCtiCtiCtiCtSA\varepsilon\varepsilon;eiCtiCtS\varepsilon\varepsilon \]
\item Utilizando (2): \[ iCtiCtiCtiCtSA\varepsilon\varepsilon;eiCtiCtS\varepsilon\varepsilon \rightarrow iCtiCtiCtiCtS;eS\varepsilon\varepsilon;eiCtiCtS\varepsilon\varepsilon \]
\item Utilizando (1): \[ iCtiCtiCtiCtS;eS\varepsilon\varepsilon;eiCtiCtS\varepsilon\varepsilon \rightarrow iCtiCtiCtiCtS;eiCtSA\varepsilon\varepsilon;eiCtiCtS\varepsilon\varepsilon \]
\item Utilizando (2): \[ iCtiCtiCtiCtS;eiCtSA\varepsilon\varepsilon;eiCtiCtS\varepsilon\varepsilon \rightarrow iCtiCtiCtiCtS;eiCtS;eS\varepsilon\varepsilon;eiCtiCtS\varepsilon\varepsilon \]
\item Utilizando (1): \[ iCtiCtiCtiCtS;eiCtS;eS\varepsilon\varepsilon;eiCtiCtS\varepsilon\varepsilon \rightarrow iCtiCtiCtiCtiCtSA;eiCtS;eS\varepsilon\varepsilon;eiCtiCtS\varepsilon\varepsilon \]
\item Utilizando (2): \[ iCtiCtiCtiCtiCtSA;eiCtS;eS\varepsilon\varepsilon;eiCtiCtS\varepsilon\varepsilon \rightarrow 
iCtiCtiCtiCtiCtS\varepsilon;eiCtS;eS\varepsilon\varepsilon;eiCtiCtS\varepsilon\varepsilon \]
\item Utilizando (1): \[ iCtiCtiCtiCtiCtS\varepsilon;eiCtS;eS\varepsilon\varepsilon;eiCtiCtS\varepsilon\varepsilon \rightarrow 
iCtiCtiCtiCtiCtiCtSA\varepsilon;eiCtS;eS\varepsilon\varepsilon;eiCtiCtS\varepsilon\varepsilon \]
\item Utilizando (2): \[ iCtiCtiCtiCtiCtiCtSA\varepsilon;eiCtS;eS\varepsilon\varepsilon;eiCtiCtS\varepsilon\varepsilon \rightarrow  \] \[ iCtiCtiCtiCtiCtiCtS\varepsilon\varepsilon;eiCtS;eS\varepsilon\varepsilon;eiCtiCtS\varepsilon\varepsilon \]
\item Utilizando (1): \[ iCtiCtiCtiCtiCtiCtS\varepsilon\varepsilon;eiCtS;eS\varepsilon\varepsilon;eiCtiCtS\varepsilon\varepsilon \rightarrow  \] \[ iCtiCtiCtiCtiCtiCtiCtSA\varepsilon\varepsilon;eiCtS;eS\varepsilon\varepsilon;eiCtiCtS\varepsilon\varepsilon \]
\item Utilizando (2): \[ iCtiCtiCtiCtiCtiCtiCtSA\varepsilon\varepsilon;eiCtS;eS\varepsilon\varepsilon;eiCtiCtS\varepsilon\varepsilon \rightarrow \] \[ iCtiCtiCtiCtiCtiCtiCtS\varepsilon\varepsilon\varepsilon;eiCtS;eS\varepsilon\varepsilon;eiCtiCtS\varepsilon\varepsilon \]

\end{enumerate}

\vspace{1em}
Finalmente, retiramos los $ \varepsilon $ y obtenemos: \[ iCtiCtiCtiCtiCtiCtiCtS;eiCtS;eS;eiCtiCtS \]

\subsection{Representación de la expresión completa}

<statement> ::= if <condition> then if <condition> then if <condition> then if <condition> then if <condition> then if <condition> then if <condition> then ; else if <condition> then <statement>; else <statement>;  else if <condition> then if <condition> then <statement>;\\

Con esto, podemos representarlo en código. En este caso se utilizo código C con su respectiva regla para evitar las ambigüedades en las derivaciones.

\lstset{language=C, breaklines=true, basicstyle=\footnotesize}
\begin{lstlisting}[frame=single]
if(condition){
    if(condition){
        if(condition){
            if(condition){
                if(condition){
                    if(condition){
                        if(condition){
                            statement;
                        }
                        else{
                            if(condition){
                                statement;
                            }
                            else{
                                statement;
                            }
                        }
                    }
                    else{
                        if(condition){
                            if(condition){
                                statement;
                            }
                        }
                    }
                }
            }
        }
    }
}
\end{lstlisting}


\end{document}





